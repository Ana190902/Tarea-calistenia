\documentclass{article}
\usepackage[utf8]{inputenc}
\usepackage[spanish]{babel}
\usepackage{listings}
\usepackage{graphicx}
\graphicspath{ {images/} }
\usepackage{cite}

\begin{document}

\begin{titlepage}
    \begin{center}
        \vspace*{1cm}
            
        \Huge
        \textbf{Tarea calistenia}
            
        \vspace{0.5cm}
        \LARGE
        informatica II
            
        \vspace{1.5cm}
            
        \textbf{Ana María Ardila Ariza}
            
        \vfill
            
        \vspace{0.8cm}
            
        \Large
        Despartamento de Ingeniería Electrónica y Telecomunicaciones\\
        Universidad de Antioquia\\
        Medellín\\
        Marzo de 2021
            
    \end{center}
\end{titlepage}

\tableofcontents
\newpage
\section{Sección introductoria}\label{intro}
Para apoyar y enseñar acerca de las instrucciones y las ambiguedades que se pueden ocasionar al darlas a otro sujeto o maquina que no sabe a donde quiere llegar, se realizaron estas instrucciones, con un estado inicial dado en clase en el cual se implementaron dos tarjetas de un mismo tamaño y una hoja de papel; y con esto llegar a un estado final ya dicho.

\subsection{Citación}
\textbf{main.tex}.

\subsection{Instrucciones}
Ponga el dedo índice de la mano derecha encima de la hoja, sin levantarlo, arrastre la hoja hacia la derecha hasta dejar a las tarjetas visibles y sin dejar caer la hoja de la superficie donde se arrastre, separe el dedo de la hoja, *con la misma mano agarre ambas tarjetas alineadas, sin soltar las tarjetas póngalas en forma vertical encima del centro de masa de la hoja, sin soltarlas, ubique el dedo pulgar en la esquina superior derecha de la tarjeta y el dedo índice en la esquina superior izquierda de la misma tarjeta (no tiene que haber mucha presión en el agarre), ubique dos dedos de la misma mano en uno de los dos lados horizontales de la tarjeta mas cercana a la palma de la mano que se está usando, sin separar los dedos y de manera paciente y despaciosa, vaya atrayendo la tarjeta al lado contrario de la otra sin separarla de la hoja, esto hasta que ambas tarjetas estén separadas dos falanges del dedo índice aproximadamente, retire lentamente los dos dedos puestos en uno de los dos lados horizontales de la tarjeta más cercana a la palma de la mano usada, finalmente con los dedos pulgar e índice puestos en las esquinas superiores de las tarjetas, vaya buscando un equilibrio entre las tarjetas manteniendo la forma triangular que tienen, si se caen, repita desde el “*” en las instrucciones.


\bibliographystyle{IEEEtran}
\bibliography{references}

\end{document}
